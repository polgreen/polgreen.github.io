%%%%%%%%%%%%%%%%%%%%%%%%%%%%%%%%%%%%%%%%%
% "ModernCV" CV and Cover Letter
% LaTeX Template
% Version 1.11 (19/6/14)
%
%
%%%%%%%%%%%%%%%%%%%%%%%%%%%%%%%%%%%%%%%%%

%----------------------------------------------------------------------------------------
%	PACKAGES AND OTHER DOCUMENT CONFIGURATIONS
%----------------------------------------------------------------------------------------

\documentclass[10pt,a4paper,times]{moderncv} % Font sizes: 10, 11, or 12; paper sizes: a4paper, letterpaper, a5paper, legalpaper, executivepaper or landscape; font families: sans or roman

\moderncvstyle{classic} % CV theme - options include: 'casual' (default), 'classic', 'oldstyle' and 'banking'
\moderncvcolor{blue} % CV color - options include: 'blue' (default), 'orange', 'green', 'red', 'purple', 'grey' and 'black'

\usepackage{lipsum} % Used for inserting dummy 'Lorem ipsum' text into the template

\usepackage[scale=0.8]{geometry} % Reduce document margins
%\setlength{\hintscolumnwidth}{3cm} % Uncomment to change the width of the dates column
%\setlength{\makecvtitlenamewidth}{10cm} % For the 'classic' style, uncomment to adjust the width of the space allocated to your name

%----------------------------------------------------------------------------------------
%	NAME AND CONTACT INFORMATION SECTION
%----------------------------------------------------------------------------------------

\firstname{Dr Elizabeth} % Your first name
\familyname{Polgreen} % Your last name

% All information in this block is optional, comment out any lines you don't need


% \email{elizabeth.polgreen@ed.ac.uk}


%----------------------------------------------------------------------------------------

\begin{document}

\makecvtitle % Print the CV title

%----------------------------------------------------------------------------------------
%	EDUCATION SECTION
%----------------------------------------------------------------------------------------

\section{Education}

\cventry{2016 -- 2020}{PhD}{Computer Science}{The University of Oxford}{}{}  

\cventry{2010 -- 2011}{Masters}{Electrical and Electronic Engineering}{The University of Cambridge}{}{}  
\cventry{2007 -- 2010}{Bachelor of Arts}{Electrical and Electronic Engineering}{The University of Cambridge}{}{}
%\cvitem{A-Levels}{4 As in Mathematics, Further Mathematics, Physics and Chemistry}

\section{Employment}
\cventry{2020 -- \hspace{3em}}{Lecturer (Assistant Professor)}{Laboratory for Foundations of Computer Science}{University of Edinburgh}{}
{}
% \cventry{2020 -- \hspace{3em}}{Volunteer Researcher}{Computer Science}{The University of California, Berkeley}{}
% {}
\cventry{2019 -- 2020}{Visiting Research Scholar}{Computer Science}{The University of California, Berkeley}{}
{}
\cventry{Jun 2018 -- Sep 2018}{Software Development Intern}{Amazon Web Services, Dresden}{}{}{Applying formal verification techniques to C code for an x86 hypervisor}

\cventry{Aug 2017 -- Oct 2017}{Software Development Intern}{Amazon Web Services, Dresden}{}{}{Development of analysis tools based on formal methods for hot-patching an x86 hypervisor}

\cventry{Sep 2015 -- Mar 2016}{Research Assistant in Verification}{Department of Computer Science, University of Oxford}{}{}{Application of machine learning techniques in verification.}

\cventry{Sep 2013 -- Aug 2015}{Research Support}{Department of Computer Science, University of Oxford}{}{}{Research support for research projects within the Systems Verification and Validation group.}

%------------------------------------------------

\cventry{Jan 2013 -- Aug 2013}{Electronics and Software Engineer}{Peach Innovations}{Cambridge}{}{Manufacture, testing and debugging of real-time rowing instrumentation systems.}

%------------------------------------------------

\cventry{Aug 2011 -- Jan 2013}{Electronics and Software Engineer}{Eg Technology}{Cambridge}{}{Embedded software and circuit design for a variety of commerical and research products.}

%----------------------------------------------------------------------------------------
%	WORK EXPERIENCE SECTION
%----------------------------------------------------------------------------------------
\section{Selected Publications}

\cventry{2022}{UCLID5: Multi-Modal Formal Modeling, Verification, and Synthesis}{ E.~Polgreen,
K.~Cheang, P.~Gaddamadugu, A.~Godbole, K.~Laeufer, S.~Lin, Y.~Manerkar, F.~Mora, S.A.~Seshia}{Computer Aided Verification (CAV)}{}{}
\cventry{2022}{Synthesis and Satisfiability Modulo Oracles}{E.~Polgreen, A.~Reynolds, S.A.~Seshia}{International Conference on Verification, Model Checking, and Abstract Interpretation(VMCAI)}{}{}

\cventry{2020}{{Using model checking tools to triage the severity of security bugs in the Xen hypervisor}}{
	B.~Cook, B.~Doebel, D.~Kroening, N.~Manthey, M.~Pohlack, E.~Polgreen, M.~Tautschnig, P.~Wieczorkiewicz
}{Formal Methods in Computer-Aided Design (FMCAD)}{}{}

\cventry{2018}{CounterExample Guided Inductive Synthesis Modulo Theories}{A.~Abate, C.~David, P.~Kesseli, D.~Kroening, E.~Polgreen}{Computer Aided Verification (CAV)}{}{}


\cventry{2017}{Automated Formal Synthesis of Digital Controllers for State-Space Physical Plants}{A.~Abate, I.~Bessa, D.~Cattaruzza, L.~Cordeiro, C.~David, P.~Kesseli, D.~Kroening, and E.~Polgreen} {Computer Aided Verification (CAV)}{}{}



% \section{Invited Talks}

% \cventry{2022}{Beyond Counterexamples: Synthesis Modulo Oracles}{}{11th Workshop on Synthesis}{}{}
% \cventry{2022}{UCLID5: multi-modal modeling, synthesis and verification}{}{3rd Workshop on Democratizing Software Verification}{}{}

% \cventry{2022}{Model checking Xen}{}{Verified Software Workshop, Isaac Newton Institute for Mathematical Sciences}{}{}
% \cventry{2021}{CounterExample Guided Inductive Synthesis Modulo Theories}{}{The Simons Institute for the Theory of Computing}{}{}


\section{Service}

\cventry{}{Steering committee}{}{ETAPS} {2022-onwards}{}
\cventry{}{Workshop chair}{}{International Workshop on Synthesis (SYNT)} {2021}{}
\cventry{}{Program committees}{}{}{}{
2022: SMT workshop, SAT, CAV, FMCAD, QEST.
2021:
 CAV (artefact evaluation), TACAS (artefact evaluation), FMCAD, QEST. 2019: International Workshop on Synthesis (SYNT)   }
{}{}{}{}
% \cventry{}{Non-program committee reviews}{}{}{}{Acta Informatica, Transactions on Programming Languages and Systems, Robotics: Science and Systems 2017, CAV 2021, SOFSEM-FOCS2017,  QEST 2017, QEST 2016, Information and Software Technology, 13th International Workshop on Discrete Event Systems}{}{}{}{}




%\section{Masters Project}

%\cvitem{Title}{\emph{Development of a data transmission system for remote robotic control}}
%\cvitem{Supervisors}{Professor Tim Flack}
%\cvitem{Description}{This project covers the development of a complete wireless data transmission system in order to provide remote control of a robotic arm. The merits of different modulation, encoding and transmission technologies were considered and key digital and analogue components of the system were simulated using open source simulation software}



% \section{Teaching}
% \cventry{2022}{Introduction to SAT/SMT}{Scottish Programming Languages and Verification Summer School}{Lecturer}{}{}

% \cventry{2022}{System Design Project}{University of Edinburgh}{Lecturer}{}{}
% \cventry{2021}{Formal Verification}{University of Edinburgh}{Course organiser}{}{}
% \cventry{2022}{Reasoning and Agents}{University of Edinburgh}{Tutorials}{}{}
% \cventry{2021}{Discrete Maths and Probability}{University of Edinburgh}{Tutorials}{}{}
% \cventry{2020}{Formal Methods: Specification, Verification, and Synthesis}{UC Berkeley}{Guest Lectures}{}{}
% \cventry{2018}{Computer Aided Verification Course}{University of Oxford}{Guest Lecture}{}{}

% \cventry{2020}{Outreach Volunteer}{Lawrence Hall of Science, UC Berkeley}{}{}{}

% \section{Students}
% % \cventry{2022--}{Portfolio solving for Syntax-Guided Synthesis}{Universty of Edinburgh}{2021}{}{}

% \cventry{2022-present}{Advances and applications of synthesis}{}{Yixuan Li}{PhD candidate}{Universty of Edinburgh}

% \cventry{2021-present}{Portfolio solving for Syntax-Guided Synthesis}{Universty of Edinburgh}{}{Masters in Informatics}{Lewis Frampton}

% \cventry{2020-2021}{Online-learning for SMT-solver algorithm selection}{UC Berkeley}{}{Undergraduate project}{Nikhil Pimpilkhare}{}
% \cventry{2020-2021}{Metagrammars for syntax-guided synthesis}{UC Berkeley}{Undergraduate research project}{}{Nicholas Chan}
% \cventry{2017-2018}{CounterExample Guided Neural Synthesis}{University of Oxford}{MSc project}{}{Ralph Abboud}


% \section{Other skills and interests}
% \cventry{2010 -- 2011}{President of Cambridge University Women's Boat Club}{}{}{}{I lead a team of 30 athletes to compete at national and international events. I worked with the executive committee and the 
% coaching team to ensure smooth day-to-day running of a high-performing club.}
% \cventry{2017 -- 2019}{Marathon Coach - Oxford University Canoe Kayak Club}{}{}{}{I run weekly flat-water kayaking sessions for Oxford University students, and organise the annual varsity race} 
% %\cventry{2019}{GB Canoe Marathon}{}{}{}{I represented Great Britain in 2019 at the canoe marathon world cup}{}{}{}
% \cventry{2011 -- 2014} {Elite lightweight rower}{}{}{}{Four-time British Champion. I represented England in 2012 and 2014.}





%----------------------------------------------------------------------------------------
%	LANGUAGES SECTION
%----------------------------------------------------------------------------------------

%\section{Languages}

%\cvitemwithcomment{English}{Mothertongue}{}
%\cvitemwithcomment{French}{GCSE A*}{}
%\cvitemwithcomment{German}{GCSE A*}{}

%----------------------------------------------------------------------------------------
%	INTERESTS SECTION
%----------------------------------------------------------------------------------------

%\section{Interests}

%\renewcommand{\listitemsymbol}{-~} % Changes the symbol used for lists
%\cvitem{Rowing}{Multiple national champion between 2011 and 2014. President of the university team 2010-11}


%----------------------------------------------------------------------------------------

\end{document}